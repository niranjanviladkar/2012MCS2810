\chapter{Introduction}

\label{ch1_INTRO}

%Replace \lipsum with text.
% You may have as many sections as you please. This is just for reference.





Human activity recognition is an area of interest in the field of large scale surveillance.
A highly accurate and precise system would obviate the need for human attention
at each of the surveillance video output all the time. The system can recognize the
actions being performed by the human in the video and can decide for itself if the actions are abnormal;
in which case, it may raise an alarm for human attention.

This project tries to improve the existing activity recognition system
by adding domain knowledge and support for uncertainty to it.


\section{Motivation}
Existing activity recognition systems \cite{actionsInContext,Realistic,improving}  do not take advantage of underlying domain.
Domain may have significant influence over the activity types, the objects appearing
in the activity etc. For example, presence of a dining table improves the chances
of activity eating as compared to driving. Or if, the video is occluded partially,
but objects like bottle and chair are visible, then also, chances of activity eating
are more as compared to other activities. This relationship between objects and activities
can be captured by First Order Logic (FOL).

But merely using FOL is not sufficient. In its basic form, FOL uses hard constraints.
To explain this with example, consider a rule
\begin{equation}
	\label{MLNRule}
	\forall clip ~ HasObject( clip, bottle ) \implies HasActivity( clip, eating )
\end{equation}

In real world scenario, it is perfectly possible that a video clip contains a bottle,
but the activity might be different than eating. 
The rule gets violated in this scenario as it is universally quantified over all the clips. 
Rule may also get violated in case of noisy training data. 
Thus, using hard constraints is neither practical nor robust for real world video clips. 

Solution to this is to attach weights with rules like \eqref{MLNRule} with weights being proportional to the real world relevance.
This idea can be captured using Markov Logic Networks (MLNs). Markov Logic allows us 
to capture the relationship between activities and objects using weighted first order logic formulae.
Thus in case of noisy training data or in case of occlusion of part of the objects,
such model performs better as compared to the models built only on low level features.


This project captures the semantic relationship between activities and objects 
as well as makes the system more robust towards noisy training data.
Aim of this project is to provide end to end recognition system to improve 
the existing human activity recognition systems
by adding domain knowledge i.e. object and people information to them.



 \section{Thesis Outline}
Chapter \ref{ch2_BACKGROUND} gives the theoretical background. It explains video classification using only video features. It explains how to extract and use the video features. 
It further describes object detection in detail and also describes the theory of Markov Logic Networks. 
Chapter \ref{ch3_RELATED} throws light on previous approaches and points out potential improvement areas.
Chapter \ref{ch4_APPROACH} explains design decisions and approach of this project.
Chapter \ref{ch5_RESULTS} shows the result comparisons.
Finally thesis concludes with chapter \ref{ch6_CONCLUSION} which notes conclusion and future work.
