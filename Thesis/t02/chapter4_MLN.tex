\chapter{Markov Logic Networks}

%Replace \lipsum with text.
% You may have as many sections as you please. This is just for reference.

As explained in \cite{MarkovLogic}, Markov Logic consists of set of First order
logic (FOL) formulae associated with weights. The FOL formulae form the structure 
of Markov Logic Networks (MLNs). Background and theory of MLNs and calculation of
weights of formula is explained in the coming sections.

\section{Markov Networks}
Markov Network is an undirected graph $G$ with its nodes as set of variables $X = (X_1, X_2,\ldots, X_n) \in \rchi $.
Apart from graph, it also consists of potential functions $\phi_k$ for each clique $k$ in the graph.
Potential function $\phi_k$ is a real valued function of the state of $k^{th}$ clique.
The joint distribution is given by

\begin{equation}
	\label{jointDist}
	P(X = x) = \frac{1}{Z}{\displaystyle \prod_{k} \phi_{k}(x_{\{k\}})}
\end{equation}

where $x_{\{k\}}$ is the state of $k^{th}$ clique;
$Z$, also known as partition function, is given by

\begin{equation}
	\label{partitionFunc}
	Z = \displaystyle \sum_{x \in \rchi} \displaystyle \prod_{k} \phi_k(x_{\{k\}}) 
\end{equation}

If clique potentials are replaced by exponentiation of weighted sum of features of state,
joint distribution can be given as

\begin{equation}
	\label{jointDistWeighted}
	P(X = x) = \frac{1}{Z} exp \left( \displaystyle \sum_i w_i f_i(x) \right)
\end{equation}

where $w_i$ is the weight of the $i^{th}$ clique and $f_i(x) \in \{0,1\}$ indicates state 
of the $i^{th}$ clique.

\section{Markov Logic}

\section{SECTION NAME}
